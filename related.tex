\section{Related Work}
\label{sec:related-work}

Waskern is built from a synthesis of ideas in networked operating systems, blockchains,
software-defined networking, software-define storage, self-sovereign
identity systems, and policy-driven archival systems.  Like SDS and SDN
systems, Waskern provide a high-level programming model that lets users
define custom data-plane behaviors at multiple implementation points in a
network.  However, it differs from
state-of-the-art SDS and SDN systems in two fundamental ways.

First, Waskern spans \textit{multiple} administrative domains,
where each domain maintains the authoritative state
of its data and independently grants limited access permissions to 
other domains.  Networks and storage providers are not trusted, and the
authoritative view of the system configuration is backed by a PoW blockchain that is not
controlled by a single entity.  This is the
opposite trust model put forth by current SDN and SDS systems like
Andromeda~\cite{andromeda} and IOFlow~\cite{ioflow},
where there is a single, large, trusted administrative domain (a set of datacenters)
that includes private storage,
networks, and hypervisors, but not the data endpoints (tenant VMs and clients).

Second, Waskern's programming model is inspired by
by the UNIX programming model, where complex flow processing is implemented by
composing simple storage programs into a pipeline (possibly spanning multiple
administrative domains).
This is a significant departure from state-of-the-art systems
like Frenetic~\cite{frenetic}, Pyretic~\cite{pyretic}, and other
OpenFlow~\cite{openflow}-based SDN systems, where data-plane programs are written in a top-down
manner and mapped into data plane switches by a controller.  This model was chosen
as a three-way compromise between providing a familiar programming model, promoting code-reuse,
and giving each administrative domain control over what code runs on
its devices.

The reason we characterize Waskern as a kernel, and not as a storage mechanism
that combines existing services, is because it is fundamentally a multi-user system where
data and ownership are intractably linked across services.
This means our design requires the system to track and enforce access controls
and authentication, which is only possible if the system is aware of its users'
identities.  This is in contrast to systems like libcloud~\cite{libcloud} and
MetaSync~\cite{metasync}, which do not address sharing data between users.

Waskern is not the only system that qualifies as a ``storage kernel''.
Systems like iRODS~\cite{irods} have existed in some form or another for
decades, and implement storage archive policy logic as pre- and post-processing steps
along the data-plane.  The key differences between these systems and Waskern is
that the former are confined to one administrative domain (but with many users), and
are implemented as middleware that must be manually integrated with the domain's
legacy storage services.  By contrast, Waskern not only crosses administrative
domains, but also makes code-reuse and storage service isolation first-class
design goals.  We explore how this leads to problems in
Section~\ref{sec:motivation}.

As a kernel, Waskern draws inspiration from the body of literature on distributed
operating systems.  Like Amoeba~\cite{amoeba}, Sprite~\cite{sprite}, and
MOSIX~\cite{mosix}, Waskern manages stateful code
execution across a network.  However Waskern takes a complementary approach to
these systems for distributing load and masking faults:  instead of instantiating remote processes
and migrating them between hosts, it re-distributes flow routes across unmoving
processes.  Users may manually stop a remote process and start it up somewhere
else, and Waskern ensures that doing so does not disrupt running I/O flows.

Our prototype implementation is inspired from both Plan
9~\cite{plan9} and contemporary RESTful Web services.
It uses filesystem abstraction for representing
application state, and enforces administrative control of files' states by
ensuring that only the users decide which hosts process I/O on them.
Like RESTful services, the common-case method for accessing another user's files
is to load the remote user's designated storage logic into a local execution
context, akin to loading Javascript programs from Web servers.

However, the filesystem abstraction is an
implementation detail, not the end itself--Waskern is designed to be able to
implement other interfaces.  In addition, Waskern overcomes many
limitations in conventional Web services, such as including
first-class support for querying files by directory or attribute, allowing
file owners to control which hosts process I/O on them, and allowing users to
decide where and how their file's data gets hosted independently of the
application servers.

Finally, systems like Onename~\cite{onename} and Tierion~\cite{tierion}
store user data in untrusted providers by leveraging a public blockchain for
data authenticity.  These can be thought of as special cases of Waskern.  Data records are intractably coupled
to user identities in these systems, and they implement an untrusted storage abstraction layer over
one or more commodity storage providers.  However, they do not offer a notion of
user-defined asynchronous data-processing, which is necessary to implement common but
non-trivial storage features.

