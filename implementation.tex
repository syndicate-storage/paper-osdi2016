\section{Implementation}
\label{sec:implementation}

Waskern is a real system, and is implemented in about 30,000 lines of C++ and
16,000 lines of Python. Its UGs include a FUSE filesystem interface, a suite of
shell tools, and a RESTful Node.js server. Its RGs come with logic to leverage
to several cloud storage providers, including Amazon S3, Google Drive, and
Dropbox. The AGs come with logic to interact with GenBank, M-Lab, and iRODS
deployments.

\subsection{Standardized Flow Interchange}

An ever-present engineering task in Waskern is to ensure that its components use
the same data interchange formats and protocols expected by existing services.
To do so, Waskern uses standard HTTP and URIs to identify and to transport data in both
the data-plane and control-plane, and its gateways look and behave like standard
HTTP servers.

Adhering to these standards has yielded several benefits.  For example,
we enabled a production deployment of Waskern to make
use of an unmodified instance of the Akamai~\cite{akamai} CDN to enhance read
availability from RGs and AGs.  URIs to blocks and manifests are
globally unique and refer to immutable data, so the CDN's cache-eviction
policies do not affect Waskern's correctness.

This technique is also critical to allowing Waskern to scale horizontally.
Since URIs are unique and refer to immutable data,
an RG's back-end data store never needs to address chunk write-conflicts.
In addition, this technique allows a commodity HTTP or TCP load-balancer to distribute
chunk requests across a set of AG and RG instances without requiring them to
coordinate.  In doing so, a logically-single AG or RG can be scaled up to handle
a scalable number of UGs.

Within Waskern, each I/O flow identifies the flow type, the URI, the manifest, the
logical byte-range, and the affected chunks of the file.  Waskern gateways compose
their I/O logic in a UNIX-like pipeline to process a flow, but this
extra flow information enables more kinds of useful processing models than a linear
succession of gateways.

In particular, 1-to-$N$ and $N$-to-1 flow transformations have proven useful in
practice.  For example, in our social media alternative, Bob's RG
converts a single write I/O flow into $N$ I/O flows, each one of which
contains an erasure code to replicate.  Bob's RG may push these flows directly
to cloud storage, or even to other downstream RGs he owns for subsequent processing.

This also allows Bob to change the view of his Facebook-hosted data via his UG.
His UG reads his data from his AG, and applies the write by saving only the delta.
Then, the RG Bob gave to Alice will construct an
$N$-to-1 read flow to consume both Bob's Facebook data from his AG and his delta
from his RG and merge them into a single read-flow for her UG.

\subsection{Operationalization}

To help manage Waskern at scale, we built an automount service that allows
users to specify a ``provisioning plan'' that describes which volumes and
gateways should be made available on the hosts they control. The automount client manages the
details of provisioning, starting and stopping the correct gateways, and polling
for changes in the provision plan.  Users have the option to manage their provisioning plans
via an interactive program, to lower the initial learning curve.

By default, the provisioning plan is hosted in the user's self-sovereign
identity service profile.  This means that the automount client only needs to know
the username to look up, and guarantees that the client always
fetches the latest authentic provisioning plan without having to trust it or its
storage providers.  This is a key usability win, since it reduces the steps to
deploy Waskern to installing the relevant packages, and running the single
command to start the automounter.
