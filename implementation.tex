\section{Implementation}
\label{sec:implementation}

Waskern is a real system, and is implemented in about 30,000 lines of C++ and
16,000 lines of Python. Its UGs include a FUSE filesystem interface, a suite of
shell tools, and a RESTful Node.js server. Its RGs come with logic to leverage
to several cloud storage providers, including Amazon S3, Google Drive, and
Dropbox. The AGs come with logic to interact with GenBank, M-Lab, and iRODS
deployments.

\subsection{Operationalization}

To help manage Waskern at scale, we built an automount service that allows
users to specify a ``provisioning plan'' that describes which volumes and
gateways should be made available on the hosts they control. The automount client manages the
details of provisioning, starting and stopping the correct gateways, and polling
for changes in the provision plan.  Users have the option to manage their provisioning plans
via an interactive program, to lower the initial learning curve.

By default, the provisioning plan is hosted in the user's self-sovereign
identity service profile.  This means that the automount client only needs to know
the username to look up, and guarantees that the client always
fetches the latest authentic provisioning plan without having to trust it or its
storage providers.  This is a key usability win, since it reduces the steps to
deploy Waskern to installing the relevant packages, and running the single
command to start the automounter.
